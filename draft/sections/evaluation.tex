%!TEX root = ../Draft.tex
\section{实验与评估}
\label{evaluation}

\subsection{具体实现}

在\ref{model}一章中,我们介绍了模型的工作原理和部分实验细节,这里将这些内容总结一下并补充一些细节。

首先,对于面部关键点检测,我们使用的超参数如表\ref{tab:esr}所示。

\begin{table}[h]
	\begin{tabularx}{\textwidth}{|X|X|X|X|X|X|c|}
		\hline
		P& T & F &  K& $N_{aug}$& $\beta$& $\kappa$\\ 
		\hline 
		400& 10 & 5  & 500 & 20 & 1000& 0.3 TupleDist\\ 
		\hline
	\end{tabularx} 
	\caption{面部关键点检测使用的超参数}
	\label{tab:esr}
\end{table}

其中,$P,T,K,F,\beta,\kappa$在之前介绍模型的时候已经介绍了具体含义,分别是选择像素点的个数,回归器的总个数,每个回归器下面蕨的个数,选取特征的个数,蕨输出的平滑参数以及在关键点周围浮动的变化成都。$N_{aug}$是为了进一步提高模型表现设置的参数,在训练模型时会对训练集进行增强。由于对于每一个样本,我们需要给定一个初始的形状,在训练时我们从样本真实的形状尽心随机选择。而给定不同的初始形状,即使是同一个图片也可以看作一个新的样本。因此,$N_aug$控制了样本增强的程度。

另一方面,为了提高效果,在进行推断的时候我们会对样本进行5次重复计算,将结果去平局作为最后的输出结果。

对于高维LBP特征,我们使用的超参数如表\ref{tab:high_dimen}所示。
\begin{table}[h]
	\begin{tabularx}{\textwidth}{|X|X|X|X|c|}
		\hline
		$N_{fp}$& L & P & S & Reduced Feature Size \\ 
		\hline 
		27 & 5 & 40  & 4 & 400\\ 
		\hline
	\end{tabularx} 
	\caption{高维LBP特征使用的超参数}
	\label{tab:high_dimen}
\end{table}

其中,$N_{fp}, L, P, S$以及降维后特征维度在之前已经有所介绍,分别表示选择的特征点个数,图形金字塔层数,区域像素大小以及划分网格数。虽然实际训练中得到的面部特征点个数为68个,但由于外围(即下巴以及脸部外轮廓)的特征不够明显,同时也由于在增加维度带来的提升比较微弱,因次选择了27个内部点(比如眼睛、鼻子和嘴)。其余的各个参数的设定也是基于类似的考量。同时,这种方法可以应用到许多中特征提取的方法上,比如SIFT或者HOG。

\subsection{效果分析}

首先,我们在