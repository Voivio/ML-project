%!TEX root = ../Draft.tex
\section{讨论}
\label{discussion}

YOLO是一个很好的上手项目,首先他的结构较为简单,但是有包含目前主流的多种结构,比如跳跃链接,批归一化层等;另一方面YOLO从最初的版本一直发展到YOLOv3,吸纳了很多计算机视觉领域的新的方法,比如使用锚框,使用多尺度的图像金字塔。因此能够较为全面地对计算机视觉领域主流的方法进行一个简要的回顾,也能很好的学习计算视觉领域常见的网络结构。虽然结构简单,但是YOLO中也需要手动对一些新定义的功能层进行实现,比如shortcut,route,以及yolo层,这也能帮助了解利用nn.Module创建新的类的过程,以及更加深入的了解了nn.ModuleList,nn.Sequential等常见容器的使用和适用场景。

另一方面,YOLO不需要很高强度的训练过程。作者在网站上已经提供了训练好的参数,可以在完成网络框架后,立即进行测试和观察效果。最后,YOLO能够给实践者一个很好的反馈,因为任务的结果仍然是一个图像,可视化性。

在本项目中,还可以对利用conda管理环境和包的管理进行练习。在这个项目的实践中,我还是学习到了很多。